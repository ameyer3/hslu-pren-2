\newacronym
  {pren1}                 % id
  {PREN 1}                % display name
  {Produktentwicklung 1}  % full acronym name

\newacronym
  {pren2}                 % id
  {PREN 2}                % display name
  {Produktentwicklung 2}  % full acronym name

\newacronym
  {elektrotechnik}                    % id
  {ET}                    % display name
  {Elektrotechnik}        % full acronym name

\newacronym
  {ai}                    % id
  {AI}                    % display name
  {Artifical Intelligence}        % full acronym name
\newacronym
  {informatik}             % id
  {I}                      % display name
  {Informatik}             % full acronym name

\newacronym
  {maschinentechnik}       % id
  {M}                      % display name
  {Maschinentechnik}       % full acronym name

\newacronym
  {orb}                                    % id
  {ORB}                                    % display name
  {Oriented FAST and Rotated BRIEF}        % full acronym name

\newacronym
  {pcb}                         % id
  {PCB}                         % display name
  {Printed Circuit Board}       % full acronym name

\newacronym
  {uart}       % id
  {UART}                      % display name
  {Universal Asynchronous Receiver Transmitter}       % full acronym name

\newacronym
  {uc}       % id
  {uC}                      % display name
  {Mikrocontroller}       % full acronym name

\newacronym
  {ftm}       % id
  {FTM}                      % display name
  {Flex Timer Module}       % full acronym name

\newacronym
  {pwm}                 % id
  {PWM}                % display name
  {Pulsweitenmodulation}  % full acronym name

\newacronym
  {iic}                 % id
  {I\textsuperscript{2}C}     % display name
  {Inter Integrated Circuit}  % full acronym name


\newacronym
  {spi}                 % id
  {SPI}                % display name
  {Serial Peripheral Interface}  % full acronym name

\newacronym
  {adc}                 % id
  {ADC}                % display name
  {Analog Digital Converter}  % full acronym name

\newacronym
  {dc}                 % id
  {DC}                % display name
  {Direct Current}  % full acronym name


\newacronym
  {led}                 % id
  {LED}                % display name
  {Lichtemittierenden Dioden}  % full acronym name


\newacronym
  {irsensor}                 % id
  {IR-Sensor}                % display name
  {Infrarot Sensor}  % full acronym name

\newacronym
  {mccar}                 % id
  {MCCAR}                % display name
  {Fahrzeug von Modul Mikrocontroller Fundamentals}  % full acronym name


\newacronym
  {lcd}                 % id
  {LCD}                % display name
  {Flüssigkristallanzeige}  % full acronym name



\newacronym
  {esp32}                 % id
  {ESP32}                % display name
  {32-Bit Mikrocontroller von Espressif}  % full acronym name

\newacronym
  {i/o}                 % id
  {I/O}                % display name
  {Inputs und Outputs}  % full acronym name

\newacronym
  {gui}                 % id
  {GUI}                % display name
  {Graphical User Interface}  % full acronym name
  \newacronym
  {tui}                 % id
  {TUI}                % display name
  {Text-Based User Interface}  % full acronym name

\newacronym
  {cli}
  {CLI}
  {Command Line Interface} 

\newacronym{pla}{PLA}{Polylactid}
  
\newacronym{abs}{ABS}{Acrylnitril-Butadien-Styrol}
  

\newacronym
  {lidar}             % id
  {Lidar}                      % display name
  {Light Detection and Ranging}             % full acronym name


\newacronym{ptfe}{PTFE}{Polytetrafluorethylen}

\newglossaryentry{orb-gloss}
{
  name = Oriented FAST and Rotated BRIEF,
  description = {
 Oriented FAST and Rotated BRIEF ist ein effizientes Verfahren zur Merkmalserkennung und -beschreibung in Bildern. Er kombiniert den FAST-Algorithmus zur Eckendetektion mit dem BRIEF-Deskriptor, wobei er Robustheit gegenüber Drehungen und Skalierungen bietet
  }
}


\newglossaryentry{tinyk22}
{
    name = Tiny K22,
    description = {
    Der Tiny K22 ist ein kompakter Mikrocontroller, der sich durch hohe Rechenleistung, geringe Stromaufnahme und vielseitige Einsatzmöglichkeiten in IoT- und eingebetteten Anwendungen auszeichnet
    }
}
\newglossaryentry{pytorch}{
    name = PyTorch,
    description = {
    PyTorch ist eine Open-Source-Library für maschinelles Lernen, die dynamische Berechnung und eine einfache Implementierung neuronaler Netze bietet. Sie wird häufig in Computer Vision eingesetzt
}
}

\newglossaryentry{opencv}{
    name = OpenCV,
    description = {
    OpenCV (Open Source Computer Vision Library) ist eine Open-Source-Library für Bildverarbeitung. Sie ermöglicht effiziente Implementierung von Algorithmen zur Analyse, Bearbeitung und Interpretation von Bildern und Videos
    }
}

\newglossaryentry{dijkstra}{
    name = Dijkstra,
    description = {
    Der Dijkstra-Algorithmus ist ein Algorithmus zur Ermittlung kürzester Wege in einem Graphen mit Kantengewichten >0
    }
}
\newglossaryentry{yolo}{
    name = YOLO,
    description = {
    YOLO (You Only Look Once) ist ein schneller Bilderkennungsalgorithmus. Die Objekte in Bildern werden in Echtzeit erkennt, indem YOLO das gesamte Bild in einem einzigen Durchlauf analysiert
    }
}

\newglossaryentry{yaml}{
    name = YAML,
    description = {
        YAML (Yet Another Markup Language, später "YAML Ain’t Markup Language") ist ein leicht lesbares Datenserialisierungsformat, das häufig zur Konfiguration von Software und zum Datenaustausch verwendet wird
    }
}

\newglossaryentry{json}{
name = JSON,
description = {
JSON (JavaScript Object Notation) ist ein leichtgewichtiges Datenformat zur strukturierten Datenübertragung, das auf einer einfach lesbaren Textdarstellung von Schlüssel-Wert-Paaren basiert und häufig in Webanwendungen eingesetzt wird
}
}

\newglossaryentry{duty-cycle}{
name = Duty Cycle,
description = {Der Duty Cycle (Tastverhältnis) gibt das Verhältnis der aktiven Zeit zur Gesamtzeit eines periodischen Signals an}
}

\newglossaryentry{epoch}{
name = Epoch,
description = {
Epochs bezeichnen in maschinellem Lernen die Anzahl der vollständigen Durchläufe durch den gesamten Trainingsdatensatz während des Trainings eines Models. Mit mehreren Epochs kann das Model die Muster in den Daten besser lernen}
}

\newglossaryentry{confusion-matrix}{
name = Confusion Matrix,
description = {
Eine Confusion Matrix ist eine Matrix-Darstellung, die die Leistung eines Klassifikationsmodells visualisiert, indem sie die tatsächlichen und vorhergesagten Klassen gegenüberstellt
}
}




\newglossaryentry{easyocr}{
name = EasyOCR,
description = {
EasyOCR ist eine OCR-Bibliothek (Optical Character Recognition), die mehrere Sprachen unterstützt und tiefes Lernen verwendet, um Text aus Bildern präzise und schnell zu extrahieren
}
}

\newglossaryentry{tesseract}{
name = Tesseract OCR,
description = {
Tesseract ist eine Open-Source-OCR-Engine (Optical Character Recognition) von Google, die Text aus Bildern oder Scans erkennt und insbesondere für gedruckte Texte gut geeignet ist
}
}