\section{Nachhaltigkeit}
\label{nachhaltigkeit}

In diesem Projekt wurde bewusst auf nachhaltiges Verhalten geachtet. Wie dies gemacht wurde, wird in den folgenden Kapiteln beschrieben.

\subsection{3R-Prinzip: Reduce, Reuse, Recycle}

Die in \acrshort{pren1} definierten Nachhaltigkeitsprinzipien, die 3 R (Reduce, Reuse, Recycle), wurden auch in der praktischen Umsetzung in PREN2 angewendet.  

 In diesem Kapitel wird erläutert wie die drei R's im Verlauf des Semesters umgesetzt wurden.

\subsubsection{Reduce}

Zur Reduktion des Ressourcenverbrauchs wurde auf papierlose Zusammenarbeit gesetzt. Die gesamte Kommunikation und Dokumentation erfolgte digital. Die Overleaf-Instanz für die LaTex-Dokumentation, welche bereits im \acrshort{pren1} verwendet wurde, wurde weiterhin auf dem bereits laufenden Server betrieben.

Viele Bauteile wie Sensoren, Steckbrettmaterialien, \acrshort{pla}-Filament oder Endschalter waren bereits im Besitz einzelner Teammitglieder und konnten direkt verwendet werden. Dadurch minimierten sich zusätzliche Bestellungen und Verpackungsmaterialien, Transportwege und Kosten konnten erfolgreich reduziert werden.

\textbf{Konstruktive Nachhaltigkeitsmassnahmen}

Ein aktiver Beitrag zur Ressourcenschonung erfolgte durch den Einsatz besonders kleiner Elektromotoren, welche eine deutlich geringere Leistungsaufnahme aufweisen. Im Vergleich zu grösseren Antrieben konnten durch den bewussten Einsatz kleiner Motoren sowohl der Energieverbrauch als auch der Materialeinsatz reduziert werden. 

Bewusst verzichtet wurde auf \acrshort{abs} (wegen Umweltbelastung, schwieriger Verarbeitung, Kosten), Stahl für grossflächige Teile (wegen Gewicht, Herstellungsaufwand, Kosten) und exotische Materialien wie Kohlefaser (wegen Kosten, Verfügbarkeit, Überdimensionierung). Stattdessen wurden umweltfreundlichere, kostengünstigere und leichter verarbeitbare Materialien wie \acrshort{pla} bevorzugt.\cite{pla}



\subsubsection{Reuse}

Im Bereich Reuse wurden zahlreiche Elemente des Prototypings erfolgreich direkt in den finalen Roboter übernommen. Dazu zählen die Räder, der Greifer und die Grundplatte. Auch Elektronikkomponenten wie der Raspberry Pi, Sensoren und die Kamera wurden teilweise aus bereits vorhandenen Beständen eingebracht und auch in \acrshort{pren2} wiederverwendet. Die modulare Bauweise erlaubte es, einzelne Bauteile mehrfach zu verwenden und flexibel anzupassen. Beispielsweise konnte die Grundplatte mehrfach verändert werden, ohne ersetzt werden zu müssen. Auch das Konzept, mehrere kleine statt einer grossen Leiterplatte zu verwenden, wurde aus Nachhaltigkeitsgründen beibehalten.

Die Elektronik Komponenten können nach der Zerlegung des Roboters mehrheitlich für weitere Projekte verwendet werden.

Die Grundplatte des Prototyps aus \acrshort{pren1} wurde so konstruiert, dass sie modular erweiterbar ist. Zusätzliche Bohrungen und Nuten konnten während der Entwicklung fortlaufend eingebracht werden, ohne dass eine neue Platte gefertigt werden musste. Die Nachbearbeitung wurde mit einer Laserschneidmaschine durchgeführt. Dadurch konnte der Materialeinsatz auf eine einzige Grundplatte beschränkt werden und Abfall wurde minimiert. Eine Übersicht der verschiedenen Bearbeitungszustände ist in Abbildung \ref{fig: Weiterentwicklung der Grundplatte} dargestellt.

\begin{figure}[H] % oder [htbp]
    \centering
    \begin{subfigure}[b]{0.45\textwidth}
        \centering
        \includegraphics[width=\linewidth]{assets/MT/Grundplatte_V0.png}
        \caption{Grundplatte V0}
    \end{subfigure}
    \hfill
    \begin{subfigure}[b]{0.45\textwidth}
        \centering
        \includegraphics[width=\linewidth]{assets/MT/Grundplatte_V1.png}
        \caption{Grundplatte V1}
    \end{subfigure}

    \vspace{0.5cm}

    \begin{subfigure}[b]{0.45\textwidth}
        \centering
        \includegraphics[width=\linewidth]{assets/MT/Grundplatte_V2.png}
        \caption{Grundplatte V2}
    \end{subfigure}
    \hfill
    \begin{subfigure}[b]{0.45\textwidth}
        \centering
        \includegraphics[width=\linewidth]{assets/MT/Grundplatte_V3.png}
        \caption{Grundplatte V3}
    \end{subfigure}
    \caption{Weiterentwicklung der Grundplatte}
    \label{fig: Weiterentwicklung der Grundplatte}
\end{figure}


\subsubsection{Recycle}

Materialien, die nicht wiederverwendet werden können, sollten
recycelt werden, damit sie in den Rohstoffkreislauf zurückgeführt werden

Durch die modulare Bauweise ist der Roboter nach Gebrauch auch wieder komplett zerlegbar. Materialien und Komponenten, welche nicht wiederverwendet werden können, können einzeln recycelt werden.

Die Komponenten aus \acrshort{pla} werden demontiert und in eine Recycling-Sammlung gegeben werden wo das Material rezykliert wird.


\subsection{Ökobilanz und Materialanalyse}

\subsubsection{Betrachtung hinsichtlich Ökobilanz}
% Beschreibung des Gesamtgewichtes und der drei grössten Material-Positionen
Das Gesamtgewicht des Fahrzeugs beträgt \textbf{1.262 Kg}. In folgender Tabelle \ref{tab:kritische-mat} werden die drei grössten Material-Positionen in Kilogramm und Prozent des Gesamtgewichtes aufgelistet, gefolgt von einer Beschreibung der Recyclingfähigkeit, Entsorgung oder Abfallbehandlung dieser Materialien.

\begin{table}[H]
    \centering
    \begin{tabular}{c c c}
    \toprule
    Material & Gewicht  & Anteil (\%)\\
    \midrule
    \acrshort{mdf} & 0.166 Kg & 13\% \\
    \acrshort{pla} & 0.252 Kg & 20\% \\
    Batterie & 0.202 Kg & 16\%   \\
    \bottomrule
    \end{tabular}
    \caption{Material-Positionen nach Gewichtsanteil}
    \label{tab:kritische-mat}
\end{table}


\textbf{\acrfull{mdf}}

\acrshort{mdf} besteht aus fein zerfaserten Holzresten, die unter hohem Druck und unter Zusatz von Leimharzen verpresst werden. Die Herstellung ist energieintensiv, da Holz zerkleinert und mit Harzen verpresst wird. Der Rohstoff Holz ist jedoch nachwachsend, was die Ökobilanz verbessert insbesondere wenn Holzabfälle verwendet werden. Emissionen durch Bindemittel (z.B. Formaldehyd) sind problematisch.\cite{support-2024}
Der Herstellungsprozess ist energieintensiv und durch den hohen Anteil an Kunstharzen ökologisch problematisch.

\textit{Recyclingfähigkeit:}  MDF ist nur bedingt recycelbar, da die Verklebung mechanische Trennung erschwert und thermische Verwertung Schadstoffe freisetzen kann.\cite{support-2024}

\textit{Entsorgung und Abfallbehandlung}: Verbrennung in speziellen Anlagen ist üblich, da MDF
nicht auf Deponien entsorgt werden sollte. Bindemittel erfordern strenge Emissionskontrollen, um Schadstoffe zu minimieren.\cite{mdf-entsorgung}

Trotz diesen Problemen wurde MDF gewählt, da es immer noch nachhaltiger ist eine MDF Platte zu verwenden, als wenn die einzelnen Teile für einen Protoyp extra gefräst wurden wären. Ebenfalls war es einfach zu bearbeiten, was dafür sorgte, dass insgesamt in einem ganzen Jahr, über \acrshort{pren1} und \acrshort{pren2} nur eine Grundplatte, die einfach angepasst werden konnte, benötigt wurde.
 
\textbf{\acrfull{pla}}

PLA Ist ein biobasierter Kunststoff, der aus nachwachsenden Rohstoffen wie Maisstärke hergestellt wird. PLA ist theoretisch biologisch abbaubar, dies jedoch nur unter industriellen Kompostierbedingungen.

\textit{Recyclingfähigkeit:} In der Praxis ist die Recyclinginfrastruktur für PLA begrenzt, weshalb es häufig verbrannt wird, obwohl PLA recycelbar ist.
Es kann in speziellen Anlagen zu neuen PLA-Produkten verarbeitet werden. \cite{pla-recycling}

\textit{Entsorgung und Abfahllbehandlung:} PLA ist biologisch abbaubar unter industriellen Kompostierungsbedingungen. In der Natur zersetzt es sich nur langsam. Entsorgung erfolgt
idealerweise in Kompostieranlagen oder durch Recycling; Verbrennung ist möglich, setzt
aber CO2 frei.\cite{pla-details}


\paragraph{Schwermetalle der Batterie}

Die LiPo-Batterie, ein schwerwiegender Bestandteil des Roboters, besteht hauptsächlich aus Lithium-Metall-Oxiden, Graphit, Elektrolyt und Kunststoff-Separator. Die Batterie ist zwar kein reines Material, wurde hier aber aufgrund ihrem hohen Gewicht und ihrer ökologischen Auswirkungen dennoch aufgelistet.\cite{litio-akkus}

Die Herstellung ist aufgrund des Abbaus sehr energieintensiv und umweltbelastend. Dabei entstehen hohe CO2-Emissionen. Es werden Schadstoffe freigesetzt und es kommt zu sozialen Konflikten. Die lange Lebensdauer kann die Ökobilanz teilweise verbessern.\cite{tezgoren}

\textit{Recyclingfähigkeit:} LiPo-Batterien sind recycelbar, aber der Prozess ist komplex und teuer. In spezialisierten Anlagen können Lithium, Kobalt und andere Metalle zurückgewonnen werden.\cite{regan-2023}


\textit{Entsorgung und Abfahllbehandlung:} LiPo-Batterien dürfen nicht im Hausmüll entsorgt
werden, da sie Brandgefahr und Umweltverschmutzung verursachen. Sie müssen in Sammelstellen oder Recyclinghöfen abgegeben werden, wo sie fachgerecht behandelt werden.

Die verwendete Batterie stammt aus dem Privatbesitz eines Teammitglieds. Dieses Mitglied wird die Batterie nach \acrshort{pren2} wieder mitnehmen und weiter verwenden. Somit musste keine extra Batterie angeschafft werden und sie muss nicht entsorgt oder recycled werden nach dem Semester.


\subsubsection{Nachhaltig-kritischste Materialien}
% Auflistung und Beschreibung der nachhaltig-kritischsten Materialien
Im Folgenden werden mindestens drei der nachhaltig-kritischsten Materialien, die im Fahrzeug verbaut sind, aufgelistet. Für jedes Material wird erläutert, warum es nicht nachhaltig ist, und es werden mögliche Massnahmen zur Vermeidung vorgeschlagen.

\begin{itemize}
    \item \textbf{Material 1:} \acrfull{mdf} \\
          \textit{Grund der mangelnden Nachhaltigkeit:} Hoher Energieaufwand und vergleichbar hoher Leimanteil.\cite{support-2024}  \\
          \textit{Vermeidungsstrategie:} Leimholzplatten mit weniger Leimanteil oder Massivholzplatten verwenden. Ökolgisch zertifiziertes MDF ist nachhaltiger. MDF war besonders für das Rapid-Prototyping geeignet und leicht verfügbar. Für ein Serienprodukt sollte andere Materialien in Betracht gezogen werden.
          
    \item \textbf{Material 2:} Lithium in der Batterie \\
          \textit{Grund der mangelnden Nachhaltigkeit:} Lithium ist umweltschädlich bei der Herstellung. Abbau und Raffinerie benötigt viel Energie und Wasser und setzt Schadstoffe frei.\cite{litio-akkus} \\
          \textit{Vermeidungsstrategie:} Lithium für Batterien ist praktisch nicht vermeidbar wenn eine Batterie benötigt wird. Der Herstellungsprozess kann jedoch umweltfreundlicher gestaltet werden. Ebenfalls konnte dieser hier vermieden werden, da keine neue Batterie angeschafft werden musste.
          
    \item \textbf{Material 3:} Kupfer in Kabel \\
          \textit{Grund der mangelnden Nachhaltigkeit:} Kupfer ist umweltschädlich bei der Herstellung. Abbau und Raffinerie benötigt viel Energie und Wasser und setzt Schadstoffe frei. \\
          \textit{Vermeidungsstrategie:} Kabel und Leitungen von Hersteller beziehen, welche ausschliesslich recycletes Kupfer verwenden.
\end{itemize}

