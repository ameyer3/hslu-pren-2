\section{Realisiertes Funktionsmuster}

Das in PREN1 ausgearbeitete Konzept konnte so wie geplant umgesetzt werden. Die Funktionalen Komponenten wurden gemäss Planung integriert und 

Vergleich  Konzeptloesung und realisiseren Funktionsmuster



\subsection{Auswertung}

In diesem Kapitel wird die Umsetzung des in \acrshort{pren1} erarbeiteten Konzept anhand der damals erstellten Anforderungsliste bewertet. Die Anforderungsliste ist im Anhang in Kapitel \nameref{anforderungliste} angehängt.

\subsubsection{Anforderungen}
Die physischen Anforderungen an den Roboter konnten vollständig erfüllt werden. Das maximal zulässige Gewicht von 2 kg wurde mit einem Gesamtgewicht von 1.262 kg, insbesondere durch den Einsatz von Leichtbaustrukturen, deutlich unterschritten. Die maximale Baugrösse 300x300x800mm kann in der Startkonfiguration mit Greifer in Ausgangsstellung eingehalten werden. Zusätzlich erforderliche Komponenten wie ein Hauptschalter, ein Eingabetaster zur Zieleingabe sowie eine Statusanzeige wurden erfolgreich integriert.

Die Anforderung 1.6 Der Greifer ist in der Lage das Hindernis mit einer Masse von ca. 300g anzuheben


Die Bilderkennung konnte mithilfe des Kameratestaufbaus aus PREN1 und einem leicht veränderten Code bereits vor Ort getestet und weiter optimiert werden. Gesperrte Knoten konnten mit bei unseren Testläufen mit einer XXXX\% Sicherheit erkennt werden. Das folgen der Linie erfolgt wie in PREN1 geplant mit mehreren Phototransistoren. 

Die Konstruktionswerkstoffe konnten wie in PREN1 vorgesehen unter Berücksichtigung der im Kapitel Nachhaltigkeit aufgeführten Kriterien ausgewählt werden.

Zur Laufzeit für den Parcour kann akutell noch keine Aussage gemacht werden da nicht alle Teilfunktionen implementiert worden sind. 



TODO HIER AUCH TESTS: AUFBAU, TRAVERSIERUNG ZEIT