\section{Realisiertes Funktionsmuster}

Das in PREN1 ausgearbeitete Konzept konnte so wie geplant umgesetzt werden. Die Funktionalen Komponenten wurden gemäss Planung integriert und 

Vergleich  Konzeptloesung und realisiseren Funktionsmuster



\subsection{Auswertung}

\subsubsection{Anforderungen}

In diesem Kapitel wird der umgesetzte Roboter bewertet anhand der in \acrshort{pren1} erstellten Anforderungsliste. Diese ist im Anhang angehängt in Kapitel \nameref{anforderungliste}.

Die auf Grundlage der Aufgabenstellung in PREN1 definierten Mindestanforderungen an den Roboter konnten vollständig erfüllt werden. Das maximal zulässige Gewicht von 2 kg wurde mit einem Gesamtgewicht von XXXXX kg, insbesondere durch den Einsatz von Leichtbaustrukturen, deutlich unterschritten. Zusätzlich erforderliche Komponenten wie ein Hauptschalter, ein Eingabetaster zur Zieleingabe sowie eine Statusanzeige wurden erfolgreich integriert.

Die Bilderkennung konnte mithilfe des Kameratestaufbaus aus PREN1 und einem leicht veränderten Code bereits vor Ort getestet und weiter optimiert werden. Gesperte Knoten konnten mit bei unseren Testläufen mit einer XXXX\% Sicherheit erkennt werden. Das folgen der Linie erfolgt wie in PREN1 geplant mit mehreren Phototransistoren. 

Die Konstruktionswerkstoffe konnten wie in PREN1 vorgesehen unter Berücksichtigung der im Kapitel Nachhaltigkeit aufgeführten Kriterien ausgewählt werden.

Zur Laufzeit für den Parcour kann akutell noch keine Aussage gemacht werden da nicht alle Teilfunktionen implementiert worden sind. 



TODO HIER AUCH TESTS: AUFBAU, TRAVERSIERUNG ZEIT