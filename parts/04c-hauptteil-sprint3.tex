%%%%%%%%%%%%%%%%%Epic 7, 8, 9, 10%%%%%%%%%%%%%%%%%%%%%%%%%%%%%%%%%%%%%%%%%%%%%%%%%%%%%%%

\subsection{Objekte erkennen}

mehrere Epics zusammengefasst 

\subsubsection{Ultraschall}
\label{ultraschall}

Backup-Hinderniserkennung mittels Ultraschallsensor

Zur Erhöhung der Betriebssicherheit ist das Fahrzeug mit einer redundanten Hinderniserkennung ausgestattet. Diese erfolgt über einen Ultraschallsensor, der kontinuierlich die Entfernung zu Objekten im unmittelbaren Umfeld misst. Der Sensor ist direkt mit dem Mikrocontroller TinyK22 verbunden.

Die Messwerte werden in Echtzeit ausgewertet. Sobald ein Objekt in einer Distanz von weniger als 10 cm erkannt wird, wird automatisch ein Not-Stopp ausgelöst, um Kollisionen zu vermeiden. Gleichzeitig wird ein definierter Fehlercode generiert und über eine serielle Schnittstelle an den Raspberry Pi übertragen, um den Fehlerstatus zu protokollieren und entsprechende Maßnahmen einleiten zu können.


\subsubsection{YOLOv11 Model trainieren}
\textbf{Todo timo}

3 Iterationen
Learn by doing
fine tuning
fehler haftes labeling
\begin{enumerate}
    \item Rohdatenset bereitstellen von Raspi cam
    \item mit roboflow vertraut machen
    \item roboflow labeling
    \item data augmentation
\end{enumerate}
testtesttest

...

\subsubsection{Modelresultate auswerten}
\label{model-results}

Um die Modelresultate auszuwerten wurde das .pt File in den Ordner der Navigation kopiert und es wurde eine ObjectDetector Klasse und ein Object Enum erstellt.

TODO UML

Diese ObjectDetector Klasse erstellt das Model aufgrund des .pt Files bei der Instanzierung eines Objektes. Diese Objekt wird aufgerufen, wenn ein Nachbarsknoten geprüft werden soll und führt einne Prozess von drei Schritten durch:

\begin{enumerate}
    \item Inference (Objekterkennung)
    \item Resultat parsen
    \item Nächstes Objekte vor dem Roboter finden.
\end{enumerate}

Im Teil der Inference wird ein Bild an das Model gegeben. Dabei sollen nur die erkannten Objekte zurückgegeben werden, die mit einer definierten prozentualen Gewissheit erkannt wurden. Diese Gewissheit ist als Konstante definiert auf TODO WERT \& WIESO.

Diese Resultate werden dann geparsed, sodass alle erkannten Objekte mir ihrer ID, mit der Confidence, mit der sie erkannt wurden, und mit ihrem Standort auf dem Bild zurückgegeben werden.

Aus dieser Liste werden nun alle Objekte betrachtet, die sich auf der Mittellinie des Bildes befinden. Somit kann sichergestellt werden, dass nicht aus Versehen Objekte, die sich nicht auf der Fahrbahn, die betrachtet wird, befinden gespeichert werden. Die Mitellinie wird berechnet aus der Breite des Bildes. Diese Objekte werden sortiert nach ihren Koordinaten auf dem Bild. Das erste Element in dieser List, ist das nächste Objekt zum Roboter. Die Id des Objektes, die vom Model zurückgegbene wird, korrespondiert mit der ID des erstellten Enums, damit das erkannte Objekt als Enum zurückgegeben wird.

In folgenden Fällen wird nicht einfach das nächste Objekt zurückgegeben:

\begin{itemize}
    \item Falls das nächste Objekt eine Barriere und das zweitnächste eine Pylone ist. In diesem Fall wird zurückgegeben, dass eine Pylone das nächste Objekt ist, da dieser Knoten sowieso nicht befahrbar ist und aus dem internen Graph entfernt werden soll.
    \item Falls kein Objekt erkannt wurde, wird ein Knoten zurückgegeben. Es ist wahrscheinlich, dass das Model ein Knoten verpasst und es ist weniger wahrscheinlich, dass eine Barriere oder ein Pylon verpasst wird. Falls dies doch geschehen ist, kann der Ultraschall trotzdem Objekte noch erkennen, falls ein unterwartetes Objekt doch auftritt. Es ist ein kleineres Problem ein Objekt zu verpassen, als sich eines einzubilden und fälschlicherweise Strecken zu entfernen. Dadurch dass immer mindestens ein Knoten zurueckgegeben wird, wird das Risiko, dass Knoten nicht erkannt werden, mitigiert.
\end{itemize}

Risk 1 mit-minimiert + ultraschall
Risk 2 minimiert
Risk 7 minimiert
Risk 12 minimierrt, kann zurueckfahren + model

Wie auf die einzelnen Objekte reagiert wird, ist in folgender Aufzählung beschrieben und ist gleich, wie in \acrshort{pren1} geplant.

\textbf{Pylonen erkennen}

Wird eine Pylone erkannt, wird der Knoten, der gerade geprüft wurde, inklusive alle Strecken dahin, aus dem internen Graphen entfernt.

\textbf{Knoten erkennen}

Wenn ein Knoten erkannt wird, dann geschieht nichts. Es wird interpretiert, dass sich kein Objekt auf diesem Weg befindet und die Strecke normal befahrbar ist.

\textbf{Barrieren erkennen}

Wir auf dem Bild eine Barriere erkannt, wird dies im internen Graphen gespeichert, indem die jeweilige Linie höher gewichtet wird, da es länger dauern wird diese zu überqueren.

\textbf{Entfernte Linien erkennen}

TODO LUKAS

\newpage
%%%%%%%%%%%%%%%%%Epic 11%%%%%%%%%%%%%%%%%%%%%%%%%%%%%%%%%%%%%%%%%%%%%%%%%%%%%%%
\subsection{Zieleingabe}

\subsubsection{Peripherie}
\label{zieleingabe}

Geht sehr schnell: 1 Minute aufstellen gemindert Risiko