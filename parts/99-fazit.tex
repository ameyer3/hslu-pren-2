\section{Schlussdiskussion}

In den folgenden Kapiteln wird zusammengefasst, was in \acrshort{pren2} bearbeitet wurde mit Ausblick auf den Wettbewerb.
Ebenfalls werden die gesammelten Erfahrungen bezüglich der Arbeiten und der Zusammenarbeit im Team beschrieben und welche Lehren daraus gezogen werden konnten.

\subsection{Erfüllung der Anforderungen}

TODO ->  kritische Wuerdigung der Arbeiten

\subsubsection{Ausblick}

TODO: Offene Punkte, Risiken und Ausblick

\textbf{Risiken}

Die folgenden Risiken bestehen noch fuer den Wettbewerb. TODO AUTFEILUNG: TO FIX, TO ACCEPT

\begin{itemize}
    \item Risiko TODO
\end{itemize}



\subsection{Lessons Learned}

Es konnten verschiedene Learnings aus Hindernissen, die angetroffen wurden, gezogen werden.

Das Konzept aus \acrshort{pren1} dient als Basis fuer den Roboterbau. Es gab Momente, in denen festgestellt wurde, dass Teile des Konzeptes nicht so funktioneren, wie geplant. Dies passierte beispielsweise bei der Messung der Distanz zu dem naechsten Knoten. Die funktioniert technsich zwar, war aber zu riskant, da der Knoten selber nicht immer genau erkannt werden kan. Deshalb wurde das Konzept hier angepasst. Die Herausforderung dabei ist es, dass das neue Konzept in das Gesamtkonzept passen muss. Wir haben gelernt mit unerwarteten Problemen auf angemessene Art umzugehen.

Ein weiteres Hinterdnis war das parallele Arbeiten. Zu Beginn hatten wir mehrere Prototypen des Roboters, die einzelne Teilfunktionen ausfuehren konnten. Ab einem gewissen Punkt reichten nur die Teilfunktionene aber nicht mehr und es wurde ebenfalls Zeit, den Roboter zusammenzubauen. Ab diesem Moment wurde vieles schwieriger und es entstanden viele Abhaengigkeiten. Zum einen musste koordiniert werden, wer wann den Roboter haben kann und zum anderen, gab es sequentielle Abhaengigkeiten. Zum Beispiel muss der Roboter zuerst fahren koennen, bevor getestet werden kann, ob er durch das Wegenetzt fahren kann. Dadurch enstanden Wartezeiten.
