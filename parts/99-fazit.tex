\section{Schlussdiskussion}

In den folgenden Kapiteln wird zusammengefasst, was in \acrshort{pren2} bearbeitet wurde mit Ausblick auf den Wettbewerb.
Ebenfalls werden die gesammelten Erfahrungen bezüglich der Arbeiten und der Zusammenarbeit im Team beschrieben und welche Lehren daraus gezogen werden konnten.

\subsection{Erfüllung der Anforderungen}

Der Roboter kann die einzelnen benötigten Funktionen umsetzen:

\begin{itemize}
    \item Der Roboter kann eine bestimmte Distanz vorwärts und rückwärts fahren.
    \item Der Roboter kann sich drehen.
    \item Der Roboter kann einer Linie folgen.
    \item Der Roboter kann anhalten, sobald er sich auf einem Knoten befindet.
    \item Der Roboter greift das Hindernis, sobald er es am Endschalter spürt.
    \item Der Roboter kann mit einem Ultraschall Objekte vor sich bemerken.
    \item Der Roboter kann ein Bild machen von einem Knoten und die Winkel der ausgehenden Linien erkennen.
    \item Der Roboter kann ein Bild machen zu einem Nachbarsknoten und erkennen, ob sich eine Pylone, eine Barriere oder nur ein Knoten dort befindet.
    \item Der Roboter kann den schnellsten Weg zum Ziel finden.
    \item Es kann ein Ziel ausgewählt werden.
    \item Der Roboter kann mit einem Notstop ausgeschaltet werden.
    \item Der Roboter kann erkennen, wenn er sich im Ziel befindet.
    \item Der Roboter kann mit einem Buzzergeräusch zeigen, dass er sich im Ziel befindet.
\end{itemize}

Die einzelnen Funktionen der Steuerung konnten jedoch noch nicht zusammengesetzt werden, das beeinhaltet auch den Wegstellprozess des Hindernisses.

Die Navigation wurde trotzdem getestet ohne ein Fahrwerk, der Kameraturm wurde an die Orte geschoben, die die Navigation sagte. Mit den Instruktionen der Navigation konnte der Graph erfolgreich traversiert werden. Auch die Mechanik konnte erfolgreich umgesetzt werden. Der Roboter fährt und dreht sich stabil, der Greifmechanismus funktioniert und alle Komponenten unterstützen die Funktionen erfolgreich.

Die nicht-funktionalen Anforderungen konnten ebenfalls erreicht werden: Die maximale Groesse wurde eingehalten und das Gewicht wurde klar unterschritten. Der Roboter ist robust gebaut und durch die Verwendung von \acrshort{pla} konnte nachhaltiger gearbeitet werden. Auch das Budget konnte eingehalten werden.

\subsubsection{Ausblick}

Die Funktionen konnten implementiert werden, jedoch muss zwingend noch daran gearbeitet werden, dass sie als System funktionieren. Die nächsten Wochen werden dafür verwendet, dies noch zu implementieren.

\textbf{Risiken}

Die folgenden Risiken aus der Risikobewertung bestehen noch für den Wettbewerb. 

\begin{itemize}
    \item Risiko 2: Knoten werden nicht erkannt
    \item Risiko 8: Hindernisse werden beim Anheben verschoben.
    \item Risiko 12: Roboter wählt falschen Pfad.
\end{itemize}

Jedoch sind die relevantesten Risiken, die es in den folgenden Wochen gilt zu beseitigen, dass die Zeit nicht mehr reichte, den Vorgang zur Hindernisbeseitigung zu implementieren und die einzelnen Steuerungsfunktionen zusammenzusetzen
Bei dem Zusammensetzen werden wahrscheinlich weitere Probleme auftreten, die innert kurzer Zeit behoben werden müssen.



\subsection{Lessons Learned}

Die während PREN2 aufgetretenen Hindernisse führten zu wichtigen Learnings, die in künftigen Projekten früher erkannt oder umgangen werden können.

Das Konzept aus \acrshort{pren1} dient als Basis für den Roboterbau. Es gab Momente, in denen festgestellt wurde, dass Teile des Konzeptes nicht so funktionieren, wie geplant. Dies passierte beispielsweise bei der Messung der Distanz zu dem nächsten Knoten. Die funktioniert technisch zwar, war aber zu riskant, da der Knoten selber nicht immer genau erkannt wurde. Deshalb wurde das Konzept hier angepasst. Die Herausforderung dabei ist es, dass das neue Konzept in das Gesamtkonzept passen muss. Wir haben gelernt mit unerwarteten Problemen auf angemessene Art umzugehen.

Ein weiteres Hindernis war das parallele Arbeiten. Zu Beginn hatten wir mehrere Prototypen des Roboters, die einzelne Teilfunktionen ausführen konnten. Ab einem gewissen Stand der Systemintegration konnte der Roboter nur noch als ganzes getestet werden. Es musste koordiniert werden wer wann welche Test durchführt. Der Test der Navigation wurde schlussendlich mit der Hilfe von der Kamerahalterung aus PREN1 durchgeführt, so konnten parallel andere Tests durchgeführt werden.

Im Verlauf des Projekts wurde deutlich, wie wichtig die Technologierecherche in \acrshort{pren1} war. Rückblickend hätte eine intensivere Recherche in bestimmten Bereichen in \acrshort{pren2} von großem Nutzen sein können. Insbesondere beim Problem mit den Encoder Motoren. Für zukünftige Projekte sollte allgemein beim auftretenden von Problemen frühzeitig parallel an der Fehleranalyse gearbeitet und gleichzeitig eine alternative Lösung erarbeitet werden. Dies Vorgehen hätte geholfen die Implementierung des Gyroskop früher in Betracht zu ziehen.

