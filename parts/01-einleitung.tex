\section{Einleitung}

In dieser Dokumentation wird die Umsetzung des im \acrfull{pren1} geplanten Konzeptes beschrieben. 
Die Dokumentation, in der das Erarbeiten des Konzeptes detailliert ist, befindet sich im elektronischen Anhang und wird, wenn nötig, im Text referenziert und alle Änderungen werden hervorgehoben.

Der Roboter soll ein Wegenetzwerk traversieren und Hindernisse darauf erkennen und umgehen. Die genaue Aufgabenstellung inklusive der Anforderungsliste befinden sich im Anhang ``\nameref{aufgabenstellung}'' und ``\nameref{anforderungliste}''. Das Projekt wird wieder in einem interdisziplinären Team durchgeführt. Das Team besteht aus Studierenden der Studiengänge \acrfull{maschinentechnik}, \acrfull{elektrotechnik}, \acrfull{informatik} und \acrfull{digitalengineering}. Alle für die Planung und Umsetzung benötigten Kompetenzen sind vorhanden.


In diesem Bericht werden zuerst die Risiken, die während \acrshort{pren1} und \acrshort{pren2} identifiziert wurden, beschrieben. Bei allen Risiken, die mit Massnahmen beim Roboterbau stark mitigiert werden konnten, wird auf die Beschreibung der Mitigation referenziert. Danach wird der Roboterbau aufgezeigt.

Im Roboterteil wird zuerst das Gesamtkonzept aufgezeigt. Danach werden die einzelnen Teilfunktionen beschrieben und wie sie umgesetzt wurden. Es werden neben der Umsetzung auch alle Tests und zusätzliche Berechnung beschrieben, sowie die Risikobewältigung. 
Als nächstes Kapitel wird das gesamte Funktionsmuster anhand der Anforderungen bewertet. Vor dem Fazit wird auf die Nachhaltigkeit eingegangen. Als Schlussdiskussion werden das Gelernte, weitere Risiken und offene Punkte beschrieben.

Im Anhang in Kapitel ``\nameref{projektplanung}'' befindet sich die gesamte Projektplanung, die pro Sprint detailliert wird. Ebenfalls gibt es am Ende einen Rückblick über das ganze Projekt hinsichtlich Planung. Auch sind die Kosten, die originale Aufgabenstellung und die Anforderungsliste im Anhang.

Die ausführlichen Testprotokolle sind auch im Anhang. Aus dem Hauptteil wird in den einzelnen Kapiteln auf das korrespondierende Testprotokollkapitel referenziert.


Das Ziel des Teams ist es, einen stabilen und funktionierenden Roboter zu haben. Dieser soll intensiv getestet sein und er sollte auf allfällige Risiken vorbereitet sein. Der Roboter wird zum Abschluss in Form eines Wettbewerbes gegen die anderen Roboter antreten. Das Ziel für dieses Team ist es, dass der Roboter erfolgreich das Wegenetz traversieren kann und das Ziel erreicht. Der benötigte Zeitaufwand wird gegenüber einer stabilen Funktionalität nachrangig priorisiert.
