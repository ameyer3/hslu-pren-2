\section{Einleitung}

In dieser Dokumentation wird die Umsetzung des im \acrfull{pren1} erarbeiteten Konzeptes beschrieben. 
Die Arbeit, in der das Erarbeiten des Konzeptes detailliert ist, befindet sich im elektronischen Anhang und wird, wenn nötig im Text referenziert und alle Aenderungen werden hervorgehoben.

Der Roboter kann ein Wegenetzwerk traversieren und Hindernisse darauf erkennen und umgehen. Die genaue Aufgabenstellung inklusive der Anforderungsliste befinden sich im Anhang TODO REF. Das Projekt wird wieder in einem interdisziplinären Team durchgeführt. Das Team besteht aus Studenten der verschiedenen Studiengänge \acrfull{maschinentechnik}, \acrfull{elektrotechnik}, \acrfull{informatik} und \acrfull{digitalengineering}. Alle für die Planung und Umsetzung benötigten Kompetenzen sind vorhanden.

In diesem Bericht werden zuerst die Risiken die waehrend \acrshort{pren1} und \acrshort{pren2} indentifiziert wurden beschrieben. Bei allen Risiken, die mit Massnahmen beim Roboterbau stark mitigiert werden konnten, wird auf die Beschriebung der Mitigation referenziert. Danach wird der Roboterbau beschrieben. Dabei werden die einzelnen Teilfunktionen beschrieben und wie sie umgesetzt wurden. Es werden neben der Umsetzung auch alle Tests und zusaetzliche Berechnung beschrieben, genau wie die Risikobewaeltung. Ebenfalls wird das Gesamtkonzept aufgezeigt mit Verweisen zu den Teilfunktionen und den einzelnen Komponenten. Als naechstes Kapitel wird das gesamte Funktionsmuster, das realisiert wurde, anhand der Anforderungen und Risiken bewertet. Vor dem Fazit wird auf die Nachhaltigkeit eingegangen. Als Schlussdiskussion werden das Gelernete, weitere Risiken und offene Punkte beschrieben. Im Anhang sind unter anderem die Projektplanung und die Kosten angehaengt.


Das Ziel ist es, einen stabilen und funktionierenden Roboter zu haben. Dieser soll intensiv getestet sein und er sollte auf allfällige Risiken vorbereitet sein. Der Roboter wird zum Abschluss in Form eines Wettbewerbes gegen die anderen Roboter antreten. Das Ziel für dieses Team ist es, dass der Roboter erfolgreich das Wegenetzt traversieren kann und das Ziel erreicht. Der benötigte Zeitaufwand wird gegenüber einer stabilen Funktionalität nachrangig priorisiert.
